\documentclass{beamer}
\usetheme{metropolis}           % Use metropolis theme
% Packages
\usepackage[spanish,mexico]{babel} % We speak español
\usepackage{booktabs} % To get nice looking tables
%\usepackage{hyperref} % Loaded by default while using the beamer class, so we have to use the next command
\hypersetup{
	bookmarks=true,          % show bookmarks bar?
	%unicode=false,          % non-Latin characters in Acrobat’s bookmarks
	%pdftoolbar=true,        % show Acrobat’s toolbar?
	%pdfmenubar=true,        % show Acrobat’s menu?
	%pdffitwindow=false,     % window fit to page when opened
	%pdfstartview={FitH},    % fits the width of the page to the window
	%pdftitle={My title},    % title
	%pdfauthor={Author},     % author
	%pdfsubject={Subject},   % subject of the document
	%pdfcreator={Creator},   % creator of the document
	%pdfproducer={Producer}, % producer of the document
	%pdfkeywords={keyword1, key2, key3}, % list of keywords
	%pdfnewwindow=true,      % links in new PDF window
	colorlinks=true,       	 % false: boxed links; true: colored links. 
	linkcolor=black,         % color of internal links (change box color with linkbordercolor)
	%citecolor=green,        % color of links to bibliography
	%filecolor=cyan,         % color of file links
	urlcolor=magenta         % color of external links. 
	% IMPORTANT:
	% Options from here (https://en.wikibooks.org/wiki/LaTeX/Hyperlinks#Customization). 
	% If colorlinks=true, then all kinds of links get colored.
	% linkcolor value was changed from red to black. Try with red to see the difference.
}
\usepackage{listings} % add lstlisting environment
\title{Plantilla para \texttt{beamer}}
\date{\today}
\author{Francisco Murphy Pérez}
\institute{Instituto de Biotecnología - Universidad Nacional Autónoma de México}
\begin{document}
	\maketitle
	\section{Primera parte}
	\begin{frame}{¿Por qué?}
		Porque$\ldots$
		\begin{itemize}
			\item Necesito una plantilla lista para usar.
			\item Tengo que aprender a usar mejor \texttt{Beamer}.
		\end{itemize}
	\end{frame}
	\begin{frame}{¿Por qué no en Rstudio/Quarto/Revealjs?}
		\begin{enumerate}
			\item Por razones secretas.
			\item Porque la alternativa no es lo suficientemente estable.
			\item Tengo experiencia en \LaTeX{}.
		\end{enumerate}
	\end{frame}
	\begin{frame}[fragile]{Requisitos}
		Se supone que el tema se ve mejor con esta \href{https://mozilla.github.io/Fira/}{fuente}.
		En Fedora, esto se resuelve al ingresar el siguiente comando en una terminal:\\
		%\begin{lstlisting}
		%sudo dnf install mozilla-fira-*
		%\end{lstlisting}
		% If in the output you get some strange characters
		% Use: lstlisting with gobble 
		%\begin{lstlisting}[gobble=6]
		%sudo dnf install mozilla-fira-*
		%\end{lstlisting}
		% OR
		% Use: lstinputlisting, without spaces
		\lstinputlisting{cmdline.sh}
	\end{frame}
	\section{Segunda parte}
	\begin{frame}{Matemáticas}
		La fórmula para obtener la distancia entre dos puntos A y B, con coordenadas \((x_1, y_1)\) y \((x_2, y_2)\), respectivamente, es 
		\begin{equation}
		d=\pm\sqrt{(x_{2}-x_{1})^2+(y_{2}-y_{1}^2)}
		\end{equation}
		
		Para obtener las coordenadas del punto medio M, con coordenadas \((x_\mathrm{m}, y_\mathrm{m})\),  entre dos puntos A y B, con coordenadas \((x_1, y_1)\) y \((x_2, y_2)\), respectivamente, tenemos que:
		\begin{align}
			x_\mathrm{m} = \frac{x_1+x_2}{2} \\
			y_\mathrm{m} = \frac{y_1+y_2}{2} 
		\end{align}
	\end{frame}
		\begin{frame}{Imágenes, tablas y dos columnas}
		\begin{columns}
			\column{0.5\textwidth}
			\begin{figure}
				\caption{Imagen tomada de \href{https://www.picpng.com/linux-unix-tux-penguin-cute-png-43298}{aquí}.}
				\centering
				\includegraphics[width=0.5\textwidth]{tux}
			\end{figure}
			\column{0.5\textwidth}
			\begin{table}[]
				\begin{tabular}{@{}llr@{}}
					\toprule
					\multicolumn{2}{c}{Item} &            \\ \midrule
					Animal     & Description & Price (\$) \\ \midrule
					Gnat       & per gram    & 13.65      \\
					& each        & 0.01       \\
					Gnu        & stuffed     & 92.50      \\
					Emu        & stuffed     & 33.33      \\
					Armadillo  & frozen      & 8.99       \\ \bottomrule
				\end{tabular}
				\caption{Ejemplo tomado de \href{https://www.tablesgenerator.com/}{acá}.}
				\label{tab:my-table}
			\end{table}
		\end{columns}
	\end{frame}
	\begin{frame}{Gráficas}
		% TODO: \usepackage{graphicx} required
		\begin{figure}[h]
			\centering
			\includegraphics[width=0.9\linewidth]{plot}
			\caption{Datos tomados del paquete \href{https://education.rstudio.com/blog/2020/07/palmerpenguins-cran/\#the-palmerpenguins-package}{palmerpenguins}.}
			\label{fig:plot}
		\end{figure}
		
	\end{frame}
	\begin{frame}{Citas}
		content...
	\end{frame}
	\begin{frame}{Bibliografía}
		content...
	\end{frame}
	\begin{frame}{Películas}
		content...
	\end{frame}
	
\end{document}